%%%%%%%%%%%%%%%%%%%%%%%%%%%%% Define Article %%%%%%%%%%%%%%%%%%%%%%%%%%%%%%%%%%
\documentclass{article}
%%%%%%%%%%%%%%%%%%%%%%%%%%%%%%%%%%%%%%%%%%%%%%%%%%%%%%%%%%%%%%%%%%%%%%%%%%%%%%%

%%%%%%%%%%%%%%%%%%%%%%%%%%%%% Using Packages %%%%%%%%%%%%%%%%%%%%%%%%%%%%%%%%%%
\usepackage{geometry}
\usepackage{graphicx}
\usepackage{amssymb}
\usepackage{amsmath}
\usepackage{amsthm}
\usepackage{empheq}
\usepackage{mdframed}
\usepackage{booktabs}
\usepackage{lipsum}
\usepackage{graphicx}
\usepackage{color}
\usepackage[spanish]{babel}
\usepackage{psfrag}
\usepackage{pgfplots}
\usepackage{bm}


% \usepackage{biblatex}
\usepackage[backend=biber,style=vancouver,]{biblatex}
\usepackage{csquotes}
\usepackage{soul,color}
\addbibresource{biblio.bib}
%%%%%%%%%%%%%%%%%%%%%%%%%%%%%%%%%%%%%%%%%%%%%%%%%%%%%%%%%%%%%%%%%%%%%%%%%%%%%%%

% Other Settings

%%%%%%%%%%%%%%%%%%%%%%%%%% Page Setting %%%%%%%%%%%%%%%%%%%%%%%%%%%%%%%%%%%%%%%
\geometry{a4paper}

%%%%%%%%%%%%%%%%%%%%%%%%%% Define some useful colors %%%%%%%%%%%%%%%%%%%%%%%%%%
\definecolor{ocre}{RGB}{243,102,25}
\definecolor{mygray}{RGB}{243,243,244}
\definecolor{deepGreen}{RGB}{26,111,0}
\definecolor{shallowGreen}{RGB}{235,255,255}
\definecolor{deepBlue}{RGB}{61,124,222}
\definecolor{shallowBlue}{RGB}{235,249,255}
%%%%%%%%%%%%%%%%%%%%%%%%%%%%%%%%%%%%%%%%%%%%%%%%%%%%%%%%%%%%%%%%%%%%%%%%%%%%%%%

%%%%%%%%%%%%%%%%%%%%%%%%%% Define an orangebox command %%%%%%%%%%%%%%%%%%%%%%%%
\newcommand\orangebox[1]{\fcolorbox{ocre}{mygray}{\hspace{1em}#1\hspace{1em}}}
%%%%%%%%%%%%%%%%%%%%%%%%%%%%%%%%%%%%%%%%%%%%%%%%%%%%%%%%%%%%%%%%%%%%%%%%%%%%%%%

%%%%%%%%%%%%%%%%%%%%%%%%%%%% English Environments %%%%%%%%%%%%%%%%%%%%%%%%%%%%%
\newtheoremstyle{mytheoremstyle}{3pt}{3pt}{\normalfont}{0cm}{\rmfamily\bfseries}{}{1em}{{\color{black}\thmname{#1}~\thmnumber{#2}}\thmnote{\,--\,#3}}
\newtheoremstyle{myproblemstyle}{3pt}{3pt}{\normalfont}{0cm}{\rmfamily\bfseries}{}{1em}{{\color{black}\thmname{#1}~\thmnumber{#2}}\thmnote{\,--\,#3}}
\theoremstyle{mytheoremstyle}
\newmdtheoremenv[linewidth=1pt,backgroundcolor=shallowGreen,linecolor=deepGreen,leftmargin=0pt,innerleftmargin=20pt,innerrightmargin=20pt,]{theorem}{Theorem}[section]
\theoremstyle{mytheoremstyle}
\newmdtheoremenv[linewidth=1pt,backgroundcolor=shallowBlue,linecolor=deepBlue,leftmargin=0pt,innerleftmargin=20pt,innerrightmargin=20pt,]{definition}{Definition}[section]
\theoremstyle{myproblemstyle}
\newmdtheoremenv[linecolor=black,leftmargin=0pt,innerleftmargin=10pt,innerrightmargin=10pt,]{problem}{Problem}[section]
%%%%%%%%%%%%%%%%%%%%%%%%%%%%%%%%%%%%%%%%%%%%%%%%%%%%%%%%%%%%%%%%%%%%%%%%%%%%%%%

%%%%%%%%%%%%%%%%%%%%%%%%%%%%%%% Plotting Settings %%%%%%%%%%%%%%%%%%%%%%%%%%%%%
\usepgfplotslibrary{colorbrewer}
\pgfplotsset{width=8cm,compat=1.9}
%%%%%%%%%%%%%%%%%%%%%%%%%%%%%%%%%%%%%%%%%%%%%%%%%%%%%%%%%%%%%%%%%%%%%%%%%%%%%%%

%%%%%%%%%%%%%%%%%%%%%%%%%%%%%%% Title & Author %%%%%%%%%%%%%%%%%%%%%%%%%%%%%%%%
\title{Apuntes de Tesis de Licenciatura - 2022}
\author{Haoyun Qin}
%%%%%%%%%%%%%%%%%%%%%%%%%%%%%%%%%%%%%%%%%%%%%%%%%%%%%%%%%%%%%%%%%%%%%%%%%%%%%%%

\begin{document}
    \begin{titlepage}
        \centering
        {\bfseries\LARGE Universidad Peruana Cayetano Heredia \par}
        \vspace{1cm}
        {\includegraphics[width=0.3\textwidth]{img/logo_escudo.png}\par}
        \vspace{1cm}
        {\scshape\Large Facultad de Ciencias y Filosofía \par}
        \vspace{1cm}
        {\bfseries\scshape\Large Comparación de modelos de aprendizaje automático químicamente interpretativos en la clasificación de moléculas de colorantes naturales de celdas solares sensibilizadas por tintes según relaciones estructura-propiedad \par}
        \vspace{1cm}
        {\itshape\Large Proyecto de tesis \par}
        \vfill
        {\Large Autor: \par}
        {\Large Bach. Dan Santivañez Gutarra\par}
        \vfill
        {\Large Asesora: \par}
        {\Large Dra. María Quintana Caceda \par}
        \vfill
        {\Large Lima - Perú \par}
        {\Large 2022 \par}
    \end{titlepage}

    \tableofcontents
    \newpage

    \begin{abstract}
        Las celdas solares son dispositivos que convierten energía lumínica en 
        energía eléctrica útil a través de fenómenos electroquímicos. El desarrollo
        de materiales sostenibles y con mayor eficiencia es un reto de investigación
        y también ambiental, ya que resolverlo conlleva explorar miles de millones de 
        compuestos y hallar dicho tipos de materiales favorecerían el consumo de 
        energías limpias. Esta exploración requiere recursos humanos y materiales que
        muchos investigadores no pueden proporcionarse, es entonces que las simulaciones
        computacionales y el enfoque dirigido por datos cobran importancia. El uso de 
        experimentos \textit{in silico} han ... 
    \end{abstract}



    \newpage
    %%%%%%%%%%%%%%%%%%%%%%%%%%%%%%%%%%%%%%%%%%%%
    \section{Introducción}
    \subsection{Problemática}
    
La dependencia de los combustibles fósiles traería consigo problemas como la inseguridad energética y el incremento de la temperatura global a causa del aumento de los gases de efecto invernadero. Las sociedades que concentran la producción de los combustibles fósiles podrían controlar geopolítica y económicamente a otras \cite{mayer2022fossil}, esta inseguridad energética se evidenció en la crísis del petróleo de 1973 \cite{vernon1976oil} y en la disputa entre Rusia y la Unión Europea por el suministro de gas en este año \cite{rodriguez2022improving}. En contraste, el problema del incremento de la temperatura global es un problema mediáticamente más discreto pero no menos alarmante. Tanto así que, en el 2018, la \textit{Intergovernmental Panel on Climate Change} (IPCC) emitió un informe sobre los impactos que causaría dicho incremento en 1.5°C con respecto a los niveles preindustriales para el 2040, que en resumidas cuentas, se prevé un detrimento crítico y sin retorno de la civilización y la biósfera \cite{guilyardi2018ipcc}. A fin de hacer frente a estos problemas, la transición hacia una matriz energética mundial donde predominen fuentes energéticas menos contaminantes y descentralizadas es la solución. 

\begin{figure}[!ht]
    \includegraphics[scale=0.85]{img/PorcentajeRenovable.png}
    \caption{Comparación del porcentaje de la cantidad de potencia consumida por tipo de energía renovable sin considerar la hidroeléctrica.
    Fuente: Our World in Data \cite{owidenergy}}
    \label{img:PorcentajeRenovable}
\end{figure}

Las fuentes renovables reunen dichas características, por lo que muchos gobiernos y organizaciones han tomado acciones para aprovecharlas. En ese marco, la \textit{International Renewable Energy Agency} (IRENA) realizó un análisis multisectorial en el 2020 donde propuso una hoja de ruta para que las energías renovables generen el 86\% de la electricidad global \cite{asmelash2020role}. En ese documento también se menciona que esta cifra no se alcanzaría sin la investigación ni el desarrollo de tecnologías que aprovechen dichas fuentes para hacerlas sostenibles y comercialmente viables.


Hace un año, la energía hidroeléctrica generó más de 4 mil tera-vatios por hora (TWh) en el mundo, lo que representó más del 50\% de la generación de todas las renovables \cite{irena2022international}. No obstante, no es posible extender la construcción de centrales hidroeléctricas dado que muchas comunidades no cuentan cuencas hidrográficas o caídas de agua que puedan aprovecharse. Sin contar esta fuente, las energías eólica y solar son las que encabezan la producción energética. Con el propósito de contextualizar el crecimiento del uso de estas fuentes renovables geográfica y temporalmente se muestra la Figura \ref{img:PorcentajeRenovable}. En ella se visualizan los porcentajes de potencia consumida provenientes de tres grupos de fuentes renovables no hídroeléctricas: La eólica, solar y otras (\textit{GeoBiomasaOtros}). Estos porcentajes se distribuyen desde el 2012 hasta el 2021 y se comparan las tendencias en el Perú, Sudamérica y el mundo. Se aprecia el aumento del consumo energético de la eólica y solar sobre las demás, y de entre estas dos la eólica es superior. No obstante, también es evidente que la energía solar ha tenido un crecimiento sostenido y esto se debe, entre otras razones, a la disponibilidad de nuevas tecnologías que la aprovechan.



Existen dos tipos de tecnlogías que aprovechan la energía solar: La termosolar y la fotovoltaica \cite{hammarstrom2012}. En la primera se usa la energía térmica de un fluído calentado por concentradores solares para generar electricidad por medio de generadores electromagnéticos. Pese a los esfuerzos por reducir los costos de construcción y mantenimiento de las centrales termosolares, solo países con alta demanda energética industrial y geografía apta pueden hacer rentables estas tecnologías \cite{xu2022concentrated}.

\begin{figure}[h!]
    \label{img:SerieTiempo}
    \includegraphics[scale=0.6]{img/SerieTiempo.png}
    \caption{Serie de tiempo de las mejores eficiencias obtenidas en la investigación de las celdas solares según la agrupación dada por NREL.
    Fuente: \textit{National Renewable Energy Laboratory} \cite{owidenergy}}
\end{figure}

En contraste a la termosolar, la fotovoltaica aprovecha directamente la radiación solar mediante fenómenos fotoeléctricos y de transporte de cargas, mimetizando la fotosíntesis. Si las plantas tienen a las células con clorofila en sus hojas como unidades generadoras de energía, en los sistemas estas serían las celdas solares. La estructura de una celda solar depende de los mecanismos de generación eléctrica y trasporte de carga, y estos a su vez, de la naturaleza de los materiales donde suceden estos mecanismos. Usar estas celdas solares para fines domésticos o industriales representaría una reducción de emisiones de $CO_2$/kWh cerca del 90\% en comparación al gas natural \cite{tawalbeh2021environmental}.


Las celdas solares llevan en el mercado más de 60 años, desde su descubrimiento en los  Laboratorios Bell en 1954 \cite{green2009path} y su primer uso comercial en el satélite Vanguard \cite{singh2013solar} en 1958, estos se han diversificado en estructura y eficiencia. Gracias al  \textit{National Renewable Energy Labotatories} (NREL)


\begin{figure}[h!]
    \includegraphics[scale=0.6]{img/SeriesEmergentes.png}
    \caption{Serie de tiempo de las mejores eficiencias obtenidas en la investigación de las tecnologías emergentes de celdas solares.
    Fuente: \textit{National Renewable Energy Laboratory} \cite{owidenergy}}
\end{figure}



Blakers cita a la 
\textit{National Renewable Energy Labotatories} (NREL) con el propósito de describir la clasificación que esta institución hace a varios tipos de celdas  según los mejores rendimientos que mostraron en sus fases experimentales \cite{blakers2013}, por otro lado Rathore y su equipo se basaron en la antigûedad comercial y madurez técnica de estas \cite{rathore2021}. Por fines prácticos se hace uso de esta última para describir las tres generaciones que engloban varios tipos de celdas así como sus características principales. 

En la primera generación están las celdas construídas a base de obleas de silicio monocristalinas y policristalinas. Las monocristalinas han alcanzado rendimientos mayores al 20\% \cite{gul2016}, no obstante su fabricación es costosa  \cite{srinivas2015review} debido al "recristalizado" de silicio por el método de Czochralski \cite{yu2019growth} para formar lingotes de monocristal que luego se cortan. Frente a esto, el uso de diferentes policristales en las celdas abarató el costo gracias a que se fabrican con procesos de enfriamiento simples, además sus eficiencias de conversión energética oscilan de entre 12-14\%, por lo que estas características le permiten ser el tipo de celda con más prevalencia en el mercado \cite{sharma2015solar}. 

En la segunda generación se encuentran las celdas en base películas delgadas de silicio amorfo (a-Si) \cite{kaur2016review}, cadmio-teluro (CdTe) \cite{bertolli2008} y cobre-indio-galio-selenio (CIGS) \cite{bagher2015types}, la característica principal de estas celdas es que son económicas en comparación con las de otras generaciones \cite{rathore2021}, sin embargo sus limitaciones son la inestabilidad de la eficiencia \cite{gul2016review} así como los impactos negativos en el ambiente y a la salud que causan sus materiales, como el cadmio por ejemplo \cite{bagher2015types}. 

Finalmente, la tercera generación agrupa a las celdas más recientes tales como aquellas basadas en concentradores, nano cristales, polímeros y sensibilizadas por tintes. Las celdas por concentradores son dispositivos rígidos que aprovechan la radiación y el calor que se generan en al incidir luz en lentes convexos \cite{bertolli2008}, si bien tiene viabilidad comercial se demandan mejoras en los materiales fotovoltaicos. Las celdas basadas en nano cristales o \textit{quantum dots} (QD) pueden capturar mejor la luz mientras suprimen fenómenos de recombinación además de tener un rendimiento estable por encima del 20\% \cite{kim2022conformal}, no obstante no son comercialmente viables porque son costosos de fabricar \cite{jean2018synthesis}. Por el contrario, las celdas de polímero se caracterizan por su flexibilidad pero sus rendimientos estan en promedio por debajo del 10\% y a su vez son inestables porque son sensibles a cambios de temperatura y presión \cite{gusain2019polymer}. Por último, las celdas sensibilizadas por tintes son dispositivos que tienen semiconductores fotosensibilizados y ensamblados a un sistema electrolítico que permite un flujo continuo de electrones \cite{suhaimi2015materials}, se caracterizan por su construcción versátil y de bajo costo gracias a la diversidad materiales de sus componentes, sin embargo sus rendimientos aún no se equiparan a las comerciales de silicio \cite{sharma2018dye}. 


Las celdas sensibilizadas por tintes (CSPT) pueden ser 

$\dots$ debido a todas estas razones, la investigación de nuevos materiales y procesos de fabricación para celdas solares sensibilizadas por tintes más costo eficientes cobran relevancia. 


Las estructura de las celdas solares 


Es claro que deben de validarse experimentalmente los resultados obtenidos por los métodos compuacionales, pero estos pueden discriminar cientos de candidat


pero dado un grado de precisión razonable  en la predicción de la propiedad del material, estos modelos computcionales aceleran el proceso de elegir mejores estructuras o moléculas candidatas con la correcta combinación de propiedades necesarias para cumplir satisfactoriamente con el propósito para las que fueron hechas \textit{Leon R. Devereux}. 




    \subsection{Justificación}
    La referencia es citada en \cite{wen2020}
    \subsection{Antecedentes}
    En los antecedentes se

\subsubsection{Internacionales}

\subsubsection{Nacionales}
    %%%%%%%%%%%%%%%%%%%%%%%%%%%%%%%%%%%%%%%%%%%%
    \section{Pregunta de investigación y objetivos}
    \subsection{Pregunta de investigación}
    \subsection{Objetivos}
    %%%%%%%%%%%%%%%%%%%%%%%%%%%%%%%%%%%%%%%%%%%%
    \section{Marco Teórico}
    \subsection{Celdas solares sensibilizadas por tintes (CSPT)}

    \subsubsection{Componentes de las CSPT}
    \subsubsection{Funcionamiento de las CSPT}
    \subsubsection{Tipos de sensibilizantes}

    \subsection{Colorantes naturales}

    \subsubsection{Características físicas y químicas}
    \subsubsection{Familias de colorantes naturales usados como sensibilizantes}

    \subsection{Enfoque de investigación direccionado por datos}

    \subsection{Modelos de aprendizaje automático}
    \subsubsection{Interpretación de modelos de aprendizaje automático}
    \input{lib/MarcoTeorico/CSST.tex}
    \input{lib/MarcoTeorico/DataDriven.tex}

    %%%%%%%%%%%%%%%%%%%%%%%%%%%%%%%%%%%%%%%%%%%
    \section{Meteriales y métodos}
    \subsection{Análisis estadístico}

\subsubsection{Pruebas de normalidad}

La verificación de la normalidad de la distribución de un conjunto de datos se 
puede obtener a través de técnicas gráficas y por pruebas. 

En primer lugar, aquellas basadas en gráficas tienen como 


    %%%%%%%%%%%%%%%%%%%%%%%%%%%%%%%%%%%%%%%%%%%
    \section{Referencias bibliográficas}
    \printbibliography

    %%%%%%%%%%%%%%%%%%%%%%%%%%%%%%%%%%%%%%%%%%%
    \section{Cronograma de trabajo}

    %%%%%%%%%%%%%%%%%%%%%%%%%%%%%%%%%%%%%%%%%%%
    \section{Presupuesto y financiamiento}



\end{document}