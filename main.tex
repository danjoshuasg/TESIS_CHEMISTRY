%%%%%%%%%%%%%%%%%%%%%%%%%%%%% Define Article %%%%%%%%%%%%%%%%%%%%%%%%%%%%%%%%%%
\documentclass{article}
%%%%%%%%%%%%%%%%%%%%%%%%%%%%%%%%%%%%%%%%%%%%%%%%%%%%%%%%%%%%%%%%%%%%%%%%%%%%%%%

%%%%%%%%%%%%%%%%%%%%%%%%%%%%% Using Packages %%%%%%%%%%%%%%%%%%%%%%%%%%%%%%%%%%
\usepackage{geometry}
\usepackage{graphicx}
\usepackage{amssymb}
\usepackage{amsmath}
\usepackage{amsthm}
\usepackage{empheq}
\usepackage{mdframed}
\usepackage{booktabs}
\usepackage{lipsum}
\usepackage{graphicx}
\usepackage{color}
\usepackage[spanish]{babel}
\usepackage{psfrag}
\usepackage{pgfplots}
\usepackage{bm}

\usepackage{biblatex}
\usepackage{csquotes}
\addbibresource{biblio.bib}
%%%%%%%%%%%%%%%%%%%%%%%%%%%%%%%%%%%%%%%%%%%%%%%%%%%%%%%%%%%%%%%%%%%%%%%%%%%%%%%

% Other Settings

%%%%%%%%%%%%%%%%%%%%%%%%%% Page Setting %%%%%%%%%%%%%%%%%%%%%%%%%%%%%%%%%%%%%%%
\geometry{a4paper}

%%%%%%%%%%%%%%%%%%%%%%%%%% Define some useful colors %%%%%%%%%%%%%%%%%%%%%%%%%%
\definecolor{ocre}{RGB}{243,102,25}
\definecolor{mygray}{RGB}{243,243,244}
\definecolor{deepGreen}{RGB}{26,111,0}
\definecolor{shallowGreen}{RGB}{235,255,255}
\definecolor{deepBlue}{RGB}{61,124,222}
\definecolor{shallowBlue}{RGB}{235,249,255}
%%%%%%%%%%%%%%%%%%%%%%%%%%%%%%%%%%%%%%%%%%%%%%%%%%%%%%%%%%%%%%%%%%%%%%%%%%%%%%%

%%%%%%%%%%%%%%%%%%%%%%%%%% Define an orangebox command %%%%%%%%%%%%%%%%%%%%%%%%
\newcommand\orangebox[1]{\fcolorbox{ocre}{mygray}{\hspace{1em}#1\hspace{1em}}}
%%%%%%%%%%%%%%%%%%%%%%%%%%%%%%%%%%%%%%%%%%%%%%%%%%%%%%%%%%%%%%%%%%%%%%%%%%%%%%%

%%%%%%%%%%%%%%%%%%%%%%%%%%%% English Environments %%%%%%%%%%%%%%%%%%%%%%%%%%%%%
\newtheoremstyle{mytheoremstyle}{3pt}{3pt}{\normalfont}{0cm}{\rmfamily\bfseries}{}{1em}{{\color{black}\thmname{#1}~\thmnumber{#2}}\thmnote{\,--\,#3}}
\newtheoremstyle{myproblemstyle}{3pt}{3pt}{\normalfont}{0cm}{\rmfamily\bfseries}{}{1em}{{\color{black}\thmname{#1}~\thmnumber{#2}}\thmnote{\,--\,#3}}
\theoremstyle{mytheoremstyle}
\newmdtheoremenv[linewidth=1pt,backgroundcolor=shallowGreen,linecolor=deepGreen,leftmargin=0pt,innerleftmargin=20pt,innerrightmargin=20pt,]{theorem}{Theorem}[section]
\theoremstyle{mytheoremstyle}
\newmdtheoremenv[linewidth=1pt,backgroundcolor=shallowBlue,linecolor=deepBlue,leftmargin=0pt,innerleftmargin=20pt,innerrightmargin=20pt,]{definition}{Definition}[section]
\theoremstyle{myproblemstyle}
\newmdtheoremenv[linecolor=black,leftmargin=0pt,innerleftmargin=10pt,innerrightmargin=10pt,]{problem}{Problem}[section]
%%%%%%%%%%%%%%%%%%%%%%%%%%%%%%%%%%%%%%%%%%%%%%%%%%%%%%%%%%%%%%%%%%%%%%%%%%%%%%%

%%%%%%%%%%%%%%%%%%%%%%%%%%%%%%% Plotting Settings %%%%%%%%%%%%%%%%%%%%%%%%%%%%%
\usepgfplotslibrary{colorbrewer}
\pgfplotsset{width=8cm,compat=1.9}
%%%%%%%%%%%%%%%%%%%%%%%%%%%%%%%%%%%%%%%%%%%%%%%%%%%%%%%%%%%%%%%%%%%%%%%%%%%%%%%

%%%%%%%%%%%%%%%%%%%%%%%%%%%%%%% Title & Author %%%%%%%%%%%%%%%%%%%%%%%%%%%%%%%%
\title{Apuntes de Tesis de Licenciatura - 2022}
\author{Haoyun Qin}
%%%%%%%%%%%%%%%%%%%%%%%%%%%%%%%%%%%%%%%%%%%%%%%%%%%%%%%%%%%%%%%%%%%%%%%%%%%%%%%

\begin{document}
    \begin{titlepage}
        \centering
        {\bfseries\LARGE Universidad Peruana Cayetano Heredia \par}
        \vspace{1cm}
        {\includegraphics[width=0.3\textwidth]{img/logo_escudo.png}\par}
        \vspace{1cm}
        {\scshape\Large Facultad de Ciencias y Filosofía \par}
        \vspace{1cm}
        {\bfseries\scshape\Large Uso de modelos de machine learning en la clasificación de moléculas de colorante según su desempeño en las celdas solares orgánicas sensibilizadas por tintes naturales \par}
        \vspace{1cm}
        {\itshape\Large Tesis \par}
        \vfill
        {\Large Autor: \par}
        {\Large Bach. Dan Santivañez Gutarra\par}
        \vfill
        {\Large Asesora: \par}
        {\Large Dra. María Quintana Caceda \par}
        \vfill
        {\Large Julio 2022 \par}
    \end{titlepage}


    \lipsum[2]

    $\mathbb{R}$
    \begin{equation}
        a^2=b^2+c^2
    \end{equation}
    \begin{equation}
        \begin{bmatrix}
            i&j&k\\
            1&2&3\\
            4&5&6
        \end{bmatrix}
    \end{equation}
    \begin{itemize}
        \item 1
        \item 2
        \item 3
    \end{itemize}
    \begin{enumerate}
        \item a
        \item b
        \item c
    \end{enumerate}

    La referencia es citada en \cite{wen2020}
    \begin{abstract}
        Las celdas solares son dispositivos que convierten energía lumínica en 
        energía eléctrica útil a través de fenómenos electroquímicos. El desarrollo
        de materiales sostenibles y con mayor eficiencia es un reto de investigación
        y también ambiental, ya que resolverlo conlleva explorar miles de millones de 
        compuestos y hallar dicho tipos de materiales favorecerían el consumo de 
        energías limpias. Esta exploración requiere recursos humanos y materiales que
        muchos investigadores no pueden proporcionarse, es entonces que las simulaciones
        computacionales y el enfoque dirigido por datos cobran importancia. El uso de 
        experimentos \textit{in silico} han ... 
    \end{abstract}

    \tableofcontents

    \newpage
    \section{Introducción}
    \subsection{Problemática}
    \subsection{Justificación}
    \subsection{Antecedentes}
    \section{Marco Teórico}
    \subsection{Celdas solares sensibilizadas por tintes}
    \section{Metodología}

    \printbibliography
\end{document}