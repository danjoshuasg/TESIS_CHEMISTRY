Muchas sociedades en está última década han estado esforzándose por dejar de depender de combustibles
fósiles y sus derivados debido a potenciales riesgos geopolíticos, económicos y ambientales. 
Si bien la energía renovable con mayor potencia consumida es la hidroeléctrica, con más de 4 mil tera-vatios por hora (TWh) en el 2021, y que su vez
es mayor que el total de potencia consumida por las otras energías renovables, cerca de 3.5 mil TWh, no es rentable la construcción de centrales 
hidroeléctricas en diversos territorios a diferencia de las termoeléctricas. En contraste a esta situación, las otras fuentes son menos costosas de implementar y sus unidades
de generación energética son modulables pero el reto actual es incrementar su potencia de generación y su adptación en el mercado. Entre dichas fuentes, resaltan la energía eólica y solar 
estas son las que más porcentaje de potencia de consumo han tenido  en el 2021 a nivel mundial, pero la energía solar es la que más crecimiento mostró sobre todo en sudamérica y el Perú
 (ver Figura \ref{img:PorcentajeRenovable}). Este hecho puede deberse a causas políticas, económicas y tecnológicas; pero hace evidente que serán una fuente renovable no hídricas preponderante en los próximos años. 

\begin{figure}[h]
    \includegraphics[scale=0.8]{img/PorcentajeRenovable.png}
    \caption{Comparación del porcentaje de la cantidad de potencia consumida por tipo de energía renovable sin considerar la hidroeléctrica.
    Fuente: Our World in Data \cite{owidenergy}}
    \label{img:PorcentajeRenovable}
\end{figure}

La energía solar es de fácil acceso para todos los terrenos habitables y gracias a los 140 mil TWh/h que el sol genera se considera una fuente casi ilimitada \cite{hammarstrom2012}.
Sin embargo, el aprovechamiento de esta energía requiere de dispositivos que deben ser capáces de capturar, transformar 
y almacenar la mayor cantidad de energía lumínica en trabajo útil en rangos de tiempo variantes a causa del tiempo atmosférico y
los estadíos solares.

La tecnología termosolar condensa el haz solar en un contenedor de un fluído de tal manera que este se caliente
y mueva un alternador eléctrico, mientras que la energía fotovoltaica convierte la radiación solar en energía electroquímica directamente 
por medio de fenómenos fotovoltaicos que se accionan gracias a materiales receptores, transformadores y transportadores de carga.
Las centrales termosolares tienen el potencial de generar miles de kilos-vatios por hora (KWh) no obstante 
existen desventajas que limitan la difusión de su uso tales como su alto costo en la instalación, mantenimiento y distribución por cableado 
eléctrico, así como la gran cantidad de energía desperdiciada en los procesos de transferencia de calor. Estos problemas están casi ausentes
en los sistemas fotovoltaicos ya que se usan paneles y modulos que facilitan su uso, sin embargo aún es un reto equiparar la eficiencia de estos 
dispositivos con los termosolares. Si bien existen tecnologías que unen la termosolar y fotovoltaica, el presente trabajo se enfoca en resolver
un problema dentro de la fotovoltaica.

Los sistemas fotovoltaicos llevan en el mercado cerca de 60 años, desde su \hl{lanzamiento por Laboratorios Bell en 1963}. El centro funcional de estos 
dispositivos son las celdas solares, las cuales consisten en placas compuestas por láminas superpuestas de materiales que pueden transportar carga 
entre electrodos gracias al cambio de potencial que sufren en presencia de radiación solar. Blakers cita a la 
\textit{National Renewable Energy Labotatories} (NREL) con el propósito de describir la clasificación que esta institución hace a varios tipos de celdas 
según los mejores rendimientos que mostraron en sus fases experimentales \cite{blakers2013}, por otro lado Rathore y su equipo se basaron en la antigûedad 
comercial y madurez técnica de estas \cite{rathore2021}. Por fines prácticos se hace uso de esta última para describir las tres generaciones que engloban
varios tipos de celdas así como sus características principales. 

En la primera generación están las celdas construídas a base de obleas de silicio monocristalinas y policristalinas, las primeras se fabrican cortando ligotes
de silicio grandes, estos lingotes de silicio "recristalizado" se hacen con el método de Czochralski \cite{yu2019growth} el cual encarece la producción \cite{srinivas2015review}. 
Por otro lado, las celdas de silicio policristalino
cortan  su rendimientos mayores al
20\% \cite{gul2016}. La segunda generación 






\begin{itemize}
    \item Cerca del \%80 de la población mundial viven de las importaciones 
     de combustibles fósiles son vulnerables a sufrir crisis energéticas
     debido a problemas geopolíticas y económicas.
     (IRENA)
    
    \item Las fuentes de energías 
\end{itemize}
