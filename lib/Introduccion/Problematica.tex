Muchas sociedades en está última década han estado esforzándose por dejar de depender de combustibles
fósiles y sus derivados debido a potenciales riesgos geopolíticos, económicos y ambientales. La energía solar
es una de las fuentes de energía alternativas de mayor interés por su fácil acceso y casi ilimitado suministro.
Sin embargo, el aprovechamiento de esta energía requiere de dispositivos que deben ser capáces de capturar, transformar 
y almacenar la mayor cantidad de energía lumínica en trabajo útil en rangos de tiempo variantes a causa del tiempo atmosférico y
los estadíos solares. La tecnología termosolar condensa el haz solar en un contenedor de un fluído de tal manera que este se caliente
y mueva un alternador eléctrico, mientras que la energía fotovoltaica convierte la radiación solar en energía electroquímica directamente 
por medio de fenómenos fotovoltaicos que se accionan gracias a materiales rececptores, transformadores y transportadores de carga.
Las desventajas de las centrales termosolares son su alto costo en la instalación, mantenimiento y distribución eléctrica, así como
la gran cantidad de energía desperdiciada en los procesos de transferencia de calor. Si bien 



\begin{itemize}
    \item Cerca del \%80 de la población mundial viven de las importaciones 
     de combustibles fósiles son vulnerables a sufrir crisis energéticas
     debido a problemas geopolíticas y económicas.
     (IRENA)
    
    \item Las fuentes de energías 
\end{itemize}
