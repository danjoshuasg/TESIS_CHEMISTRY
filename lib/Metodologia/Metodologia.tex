\subsection{Análisis estadístico}

\subsubsection{Pruebas de normalidad}

La verificación de la normalidad de la distribución de un conjunto de datos se 
puede obtener a través de técnicas gráficas y por pruebas. 

En primer lugar, aquellas basadas en gráficas tienen como 

\textbf{Data}

La cien
Actualmente, cerca del 50\% de las nuevas publicaciones de artículos son de acceso abierto y estimaciones hechas por (Ref 33 Y 34) indican que 
estás casi en su totalidad lo serán también para el 2040.

\textbf{Herramientas de libre acceso}

El software de acceso abierto es el conjunto de herramientas computacionales que permiten 
las operaciones de la ciencia de datos basada en datos. Este tipo de software adquiere 
importancia desde la aparición pública del internet y el sistema operativo LINUX en 1990, 
cuyos desarrollos fueron netamente colaborativos. Adicionalmente, el nacimiento de 
herramientas de control de versiones y lenguajes de programación de alto nivel permitieron 
el nacimiento de repositorios proyectos científicos (46), notebooks electrónicos (47), 
paquetes de simulacón y experimentos(49) y librerías de aprendizaje automático y 
ciencia de datos (50). 


\subsubsection{Extracción de datos}

\subsubsection{Preprocesamiento de datos}
\subsubsection{Descubrimiento y análisis de características}
\subsubsection{Entrenamiento y evaluación del modelo 1}
\subsubsection{Entrenamiento y evaluación del modelo 2}
\subsubsection{Entrenamiento y evaluación del modelo 3}
\subsubsection{Comparación de modelos}